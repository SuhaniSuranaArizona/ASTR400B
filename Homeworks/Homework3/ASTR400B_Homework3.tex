\documentclass{article}
\usepackage{amsmath}
\usepackage{graphicx} % Required for inserting images
\usepackage[bottom=2 cm, right=1.5cm, left=2 cm, top=1 cm]{geometry}
\usepackage{booktabs}
\title{ASTR400B Homework 3}
\author{Suhani Surana}
\date{February 2025}

\begin{document}

\maketitle


\section{Mass Breakdown of the local group}
\begin{table}[h]
    \centering
    \renewcommand{\arraystretch}{1.2} % Increase row spacing
    \begin{tabular}{lccccc}
        \toprule
        Galaxy Name & Halo Mass ($10^{12} M_{\odot}$) & Disk Mass ($10^{12} M_{\odot}$) & Bulge Mass ($10^{12} M_{\odot}$) & Total Mass ($10^{12} M_{\odot}$) & $f_{\text{bar}}$ \\
        \midrule
        MW & 1.975 & 0.075 & 0.01 & 2.06 & 0.0413\\
        M31 & 1.921 & 0.12 & 0.019 & 2.06 & 0.067 \\
        M33 & 0.187 & 0.0093 & 0 &  0.1963  & 0.047 \\
        \bottomrule
    \end{tabular}
    \caption{Mass Breakdown of the local group}
    \label{tab:galaxy_masses}
\end{table}
\subsection{Calculations} 
\textbf{{MW (Milky Way)}}

Total mass:

\[
\text{Total Mass} = 1.975 + 0.075 + 0.01 = \boxed{2.06 \times 10^{12} M_{\odot}}
\]

Baryon fraction:

\[
f_{\text{bar}} = \frac{0.075 + 0.01}{2.06} = \frac{0.085}{2.06} \approx \boxed{0.0413}
\]

\textbf{{MW31}}

Total mass:

\[
\text{Total Mass} =  1.921 + 0.12 + 0.019 = \boxed{2.06 \times 10^{12} M_{\odot}}
\]

Baryon fraction:

\[
f_{\text{bar}} = \frac{0.12 + 0.019}{2.06} = \frac{0.139}{2.06} \approx \boxed{0.067}
\]

\textbf{{MW33}}

Total mass:

\[
\text{Total Mass} = 0.187 + 0.0093 + 0 = \boxed{0.1963\times 10^{12} M_{\odot}}
\]

Baryon fraction:

\[
f_{\text{bar}} = \frac{0.0093 + 0}{0.1963} = \frac{0.0093}{0.1963} \approx \boxed{0.0474}
\]

\textbf{{Local Group}}

Total mass:

\[
\text{Total Mass} = 2.06 + 2.06 + 0.1963  = \boxed {4.3163 \times 10^{12} M_{\odot}}
\]

Baryon fraction:

\[
f_{\text{bar}} = \frac{0.085 + 0.139 + 0.0093}{4.3163} = \frac{0.2333}{4.3163} \approx \boxed{0.054}
\]

\section{Questions}
1. The total mass is exactly the same. In both of them, halo mass dominates the total mass.
\newline \newline
2. M31's stellar mass is 1.635 times MW's stellar mass. Thus, M31 would be more luminious as compared to MW.
\newline \newline
3. MW's dark matter mass is 1.02 times the M31's dark matter mass. This is surprising as M31 has a lot more stellar mass as compared to MW. Thus, it suggests they might have different rates of stellar formation in the past or different magnitudes of feedback processes.
\newline \newline
4. MW has 4.13 \% mass locked up in baryons. M31 has 6.7 \% mass locked up in baryons. M33 has 4.74 \% mass locked up in baryons.
AGM release a lot of gas into the IGM, which could potentially lower this fraction. This could also be due to other feedback processes or missing baryons.
\end{document}
